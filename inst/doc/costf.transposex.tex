\inputencoding{utf8}
\HeaderA{costf.transposex}{Cost Function Call}{costf.transposex}
\keyword{method}{costf.transposex}
%
\begin{Description}\relax
Call the cost function after transposition of the value of the point
estimate x, so that the input row vector, given by optimsimplex, is
transposed into a  column vector as required by the cost function.
\end{Description}
%
\begin{Usage}
\begin{verbatim}
  costf.transposex(x = NULL, this = NULL)
\end{verbatim}
\end{Usage}
%
\begin{Arguments}
\begin{ldescription}
\item[\code{x}] The point estimate provide as a row matrix.
\item[\code{this}] A neldermead object.
\end{ldescription}
\end{Arguments}
%
\begin{Value}
Return the value of the cost function (called by \code{neldermead.costf}.)
\end{Value}
%
\begin{Author}\relax
Author of Scilab neldermead module: Michael Baudin (INRIA - Digiteo)

Author of R adaptation: Sebastien Bihorel (\email{sb.pmlab@gmail.com})
\end{Author}
%
\begin{SeeAlso}\relax
\code{\LinkA{neldermead.costf}{neldermead.costf}}
\end{SeeAlso}
